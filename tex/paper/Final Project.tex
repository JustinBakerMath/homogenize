\documentclass[12pt]{article}

% Packages
\usepackage{amsmath}
\usepackage{amsthm}
\usepackage{amssymb}
\usepackage{blindtext}
\usepackage[english]{babel}
\usepackage{bm}
\usepackage{caption}
\usepackage[colorlinks=true,linkcolor=red,citecolor=blue,urlcolor=cyan]{hyperref}
\usepackage{dsfont}
\usepackage{enumitem}
\usepackage{fancyhdr}
\usepackage{float}
\usepackage[utf8]{inputenc}
\usepackage{hyperref}
\usepackage[margin=1in]{geometry}
\usepackage{graphicx}
\usepackage{mathrsfs}
\usepackage{mathtools}
\usepackage[final]{pdfpages}
\usepackage{subcaption}
\usepackage{tikz-cd}
\usepackage{titlesec}
\usepackage{xcolor}
\usepackage{titlesec}
\usepackage{tocloft}


% Package Use Info
\graphicspath{ {./images/} }
\makeindex
\usetikzlibrary{matrix,arrows}

% Shortcuts and Functions. 
\newcommand{\abs}{\text{abs}}
\newcommand{\bp}{\begin{pmatrix}}
\newcommand{\ep}{\end{pmatrix}}
\newcommand{\C}{\mathbb{C}}
\newcommand{\mP}{\mathcal{P}}
\newcommand{\N}{\mathbb{N}}
\newcommand{\Q}{\mathbb{Q}}
\newcommand{\R}{\mathbb{R}}
\newcommand{\Z}{\mathbb{Z}}

\newtheorem{thm}{Theorem}[section]
\newtheorem{cor}{Corollary}[thm]

\usepackage[backend=bibtex,giveninits=true,sorting=nyt,style=alphabetic,natbib=true,maxcitenames=2,maxbibnames=10,url=false,doi=true,backref=false]{biblatex}

\titleformat{\section}
  {\normalfont\scshape\center}{\thesection.}{.5em}{}
  
  \titleformat{\subsection}
  {\normalfont\scshape}{\thesubsection}{.5em}{}



\title{Optimal Transportation in Design}

\date{\today}
\author{Justin Barker}

\begin{document}

\maketitle

\begin{abstract}
The algorithms learned in convex optimization can be used to numerically solve the optimal transportation problem. The particular optimal transportation problem of interest is the Katorovich-Rubenstein (KR) optimal transportation problem for its wide adaptability to conductivity, drone swarming, etc. In this paper convex optimization methods are used to solve the 1D and 2D KR optimal transportation problem. This paper also contains the analysis of the application of these methods.
\end{abstract}


\medskip

\section{Optimal Transportation}
Optimal transportation considers the cost of transferring masses from one location to fill another location. Formally, let $X,Y$ be metric spaces with $\mu$ a measure on $X$, $\mu\in\mathcal{M}(X)$. A transportation map $T:X\rightarrow Y$ is a measurable map with the push-forward of $\mu$ by $T$ being defined by the measure $T_\#\mu$

$$\forall B\subseteq Y, T_\#\mu(B)=\mu(T^{-1}(B))$$

\noindent where $B$ is an open ball in $Y$.

The objective of the optimal transportation problem is to find a transportation map $T$ which  minimizes the following functional.

\begin{align}
I[T^*]&=\inf_{T_\#=\nu} \int_{X}c(x,T(x))d\mu(x) \label{eq1}\tag{1}
\end{align}

This problem may be reformulated as an unconstrained optimization problem by considering an indicator function for the constraint. In particular let the indicator function be defined for continuous and bounded functions $\phi,\psi$

$$\sup_{\phi,\psi} \int_X\phi d\mu + \int_Y\psi d\nu - \int_{X\times Y} \left(\phi(x)+\psi(y)\right)d\gamma = \begin{cases} 0 & \text{ if }\gamma\in\Pi(\mu,\nu)\\ +\infty & \text{otherwise}\end{cases}$$
\noindent
The  optimization problem can then be written as the following 

\begin{align*}
\inf_\gamma \int_{X\times Y} cd\gamma +\sup_{\phi,\psi} \int_X\phi d\mu + \int_Y\psi d\nu - \int_{X\times Y} \left(\phi(x)+\psi(y)\right)d\gamma
\end{align*}

\noindent
With some effort, omitted here, one can show that the inf and sup are interchangeable.

\begin{align*}
\sup_{\phi,\psi} \int_X\phi d\mu + \int_Y\psi d\nu + \inf_\gamma \int_{X\times Y} (c(x,y)-\left(\phi(x)+\psi(y)\right))d\gamma
\end{align*}

\noindent
We may similarly remove the infemum as a constraint. Notice the following

$$\inf_\gamma \int_{X\times Y} (c(x,y)-\left(\phi(x)+\psi(y)\right))d\gamma = \begin{cases} 0 & \text{ if } \phi(x)+\psi(y)\leq c(x,y)\\ -\infty & \text{otherwise} \end{cases}$$
\noindent
Therefore we have the following constrained optimization problem.

\begin{align*}
J[\phi,\psi]=&\sup_{\phi,\psi} \int_X\phi d\mu+\int_Y\psi d\nu \\
\text{s.t.} \quad & \phi(x)+\psi(y)\leq c(x,y)
\end{align*}
\noindent
This is known as the dual problem. The values of $\phi$ and $\psi$ are called Kantorovich potentials and are also Lagrange multipliers. It can be shown that the dual problem has a solution and that the problem is strongly dual. That is $K[\gamma^*]=J[\phi^*,\psi^*]$. The proof of the solution and of strong duality will also be omitted here.

In formulating the solution, one relies on the fact that the values of $\phi$ and $\psi$ are $c$ and $\bar{c}$-conjugate for a continuous $c(x,y)$. Here $c$ and $\bar{c}$-conjugate are defined as

\begin{align*}
\phi^c(y)&=\inf_{x\in X} c(x,y)-\phi(x)\\
\psi^{\bar{c}}(x)&=\inf_{y\in Y} c(x,y)-\psi(y)
\end{align*}

Furthermore if $c$ is a metric over a single compact domain $X$, then $\phi^c(y)=\inf_{x\in X} c(x,y)-\phi(x)$ is 1-Lipschitz with respect to the metric $c$. Furthermore $\phi^{c\bar{c}}$ is also 1-Lipschitz. Therefore $c(x,y)\geq \phi(x)-\phi(y)$ and

$$\phi^c(y)=\inf_{x\in X} c(x,y)-\phi(x)=-\phi(y)$$
\noindent
Therefore $\phi^c(y)=-\phi(y)$ and we have the following Kantorovich-Rubinstein optimization problem.

\begin{align*}
R[\phi]=&\sup_{\phi,\psi} \int_X\phi d(\mu-\nu) \\
\text{s.t.} \quad & |\phi(x)-\phi(y)| \leq c(x,y)
\end{align*}


\subsection{Modified Kantorovich-Rubinstein Constraints}

In the constraint of the Kantorovich-Rubinstein optimal transportation problem we have that $c(x,y)$ is a continuous metric over the domain $X$. This $c(x,y)$ is the cost to transfer some masses from the point $x$ to the point $y$. This is associated with a ``shipping cost", the cost to ship materials from one location to another.

Alternatively, one can view the cost of transferring materials through a particular point in the domain and in a particular direction. This is associated with a ``toll cost"; at each point in space the materials pay a toll to travel along a particular ``road". This cost $\delta(x,v)$ is defined as follows

$$\delta(x,v) = \lim_{y\xrightarrow{v} x} \frac{c(x,y)}{\lVert y-x\rVert}$$
\noindent
This value $\delta$ arises from applying the following equivalent modifications to the constraint.

\begin{align*}
\abs(\phi(x)-\phi(y)) &\leq c(x,y)\\
\frac{\abs(\phi(x)-\phi(y))}{\Vert y-x\Vert}&\leq \frac{c(x,y)}{\Vert y-x\Vert}\\
\lim_{y\xrightarrow{v} x}\frac{\abs(\phi(x)-\phi(y))}{\Vert y-x\Vert}&\leq \lim_{y\xrightarrow{v} x}\frac{c(x,y)}{\Vert y-x\Vert}\\
\abs\left(\lim_{y\xrightarrow{v} x}\frac{\phi(x)-\phi(y)}{\Vert y-x\Vert}\right)&\leq \lim_{y\xrightarrow{v} x}\frac{c(x,y)}{\Vert y-x\Vert}\\
\abs(D_v\phi(x))&\leq\delta(x,v),\quad \forall x,\forall v
\end{align*}

Therefore an equivalent optimization problem can be formulated provided that $\delta(x,v)\geq 0$ for all $x$ and $v$.

\begin{align*}
S[\phi]=&\sup_{\phi,\psi} \int_X\phi d(\mu-\nu) \\
\text{s.t.} \quad & \abs(D_v\phi(x))\leq\delta(x,v)
\end{align*}



\section{Convex Optimization Methods}

does it auto update??? 


\subsection{Linear Programming}


\subsection{Iterative Updating Algorithm}


\section{Numerical Results}
 

\printbibliography

\end{document}